\documentclass{article}
\usepackage{lmodern}
\usepackage[T1]{fontenc}
\usepackage{lettrine}
\usepackage[utf8]{inputenc}
\usepackage{mathtools}
\usepackage[spanish,activeacute]{babel}

\title{La suerte de Flavio Josefo}
\author{Christian Alejandro Herrejon Villa}

\begin{document}
	\maketitle
	\lettrine{E}l problema recibe el nombre de Flavio Josefo, historiador judío que vivió en el siglo primero. Según el relato del asedio de Yodfat de Josefo, él y sus 40 soldados quedaron atrapados en una cueva por los soldados romanos. Eligieron el suicidio durante la captura, y establecieron un método para suicidarse en serie por sorteo. Josefo afirma que, por suerte o, posiblemente de la mano de Dios, él y otro hombre se mantuvieron hasta el final y se rindieron a los romanos en lugar de matarse a sí mismos. Esta es la historia dada en el libro 3, capítulo 8, parte 7 de La Guerra de los Judíos de Josefo.\\
	\newline\maketitle Fórmula
	\newline La solución depende de cuanto sea el valor de \textbf{k=2}, para una persona inicial sea impar.\\
	\begin{equation}
		f(2j)=2f(j)-1
	\end{equation}
	
La solución depende de cuanto sea el valor de \textbf{k=2}, para una persona inicial par.\\	
\begin{equation}
		f(2j+1)=2f(j)+1
	\end{equation}
	
	Debido a que no comprendo de una mejor manera el problema para cuando \textbf{k!=2}, me es imposible explicar la solución y son correspondida fórmula.
	
\end{document}